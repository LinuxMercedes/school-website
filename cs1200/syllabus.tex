\documentclass{article}
\usepackage{amsmath}
\usepackage{hyperref}
\hypersetup{colorlinks=false,
allbordercolors={0 0 0},
pdfborderstyle={/S/U/W 1}
} % Set link colors throughout the document

\title{CS 1200: Discrete Math \\ Summer 2017 Syllabus}
\date{\today}
\author{Nathan Jarus}

\begin{document}
\maketitle

This course is meant as a first introduction to discrete mathematics with emphasis on its use in computer science. Topics include: Propositional and Predicate Logic, Proof Techniques, Sequences, Mathematical Induction, Set Theory, Functions and Relations, Counting and Probability, and Graph Theory.

\section{Objective}

To provide students with the mathematical foundation, level of abstract thinking, and knowledge of discrete mathematics topics essential to computer science.

\section{Instructor}
The course will be taught by \textbf{Nathan Jarus}. \\*
\textbf{E-mail}: \href{mailto:jarus@mst.edu}{jarus@mst.edu} \\*
\textbf{Course Website}: \href{https://www.mst.edu/~nmjxv3/cs1200}{https://www.mst.edu/~nmjxv3/cs1200} \\*
\textbf{Office}: Engineering Research Lab 113 \\*
\textbf{Office Hours}: Tuesday \& Thursday, 3-4 PM\\

\section{Course Information}
\textbf{Location}: Computer Science Building room 202\\*
\textbf{Time}: Weekdays, 1:50--2:50 PM\\*
\textbf{Required Text}: \textit{Discrete Mathematics with Applications, Fourth Edition}, Susanna S. Epp, Brooks/Cole, ISBN-13: 978-0-495-39132-6\\*
\textbf{Recommended Text}: \textit{Student Solutions Manual and Study Guide for the Fourth Edition of Discrete Mathematics with Applications}, Susanna S. Epp, Brooks/Cole, ISBN-13: 978-0-495-82613-2\\

\section{Prerequisites}
This course is aimed in general at freshman and sophomore students in science and engineering, and in particular at freshman students in computer science and sophomore students in computer engineering. The sole prerequisite is a "C" or better grade in Comp Sci 1570 - Introduction to Programming.

\section{Attendance}
Attendance in class is mandatory.
You will be responsible for all the material presented in class and otherwise designated (such as reading material).
Come to class on time and be prepared.
If you miss class for some reason, it is your responsibility to find out what you've missed.

\section{Grading}

The final grade for this class is broken down as follows:

\begin{itemize}
	\item Exams --- 65\%
	\item Quizzes --- 10\% 
	\item Programming Assignments --- 25\%
\end{itemize}

Grades will be assigned following a straight scale:

\begin{tabular}{r|l}
	$90-100$ & A \\
	$80-89$ & B \\
	$70-79$ & C \\
	$60-69$ & D \\
	$< 60$ & F
\end{tabular}

\subsection{Exercises}
The course schedule contains recommended exercises for each section.
You are strongly encouraged to try them out!
If you want feedback on your solutions, I am happy to look them over.
Please submit either a paper copy in person or a typeset copy via email (no pictures of paper, please).
I recommend you typeset your solutions using \LaTeX{}.

\subsection{Exams}
There will be three exams and one comprehensive final exam.
The final exam is optional; if you choose to take it, it will replace your lowest test score as well as counting for an exam in itself.
Taking the final cannot hurt your grade.

Exam scores are calculated following these formulas:

\begin{align*}
	NoFinal &= (Exam1+Exam2+Exam3)/3 \\
	WithFinal &= (Exam1+Exam2+Exam3+2*Final-Min(Exam1,Exam2,Exam3))/4 \\
	ExamGrade &= Max(NoFinal, WithFinal)
\end{align*}

\subsection{Quizzes}
Short quizzes will be given biweekly at the beginning of class.
Your lowest quiz score will be dropped.

\subsection{Programming Assignments}
Programming assignments will be submitted via Canvas (or, if you so choose, through GitLab).
All code should be properly commented and documented and must run on a modern Linux machine (such as the campus Linux machines).

Unless specified otherwise, all assignments are due at 11:59pm on their respective due dates.
The penalty for late submission is a 5\% point deduction for the first 24 hour period and a 10\% point deduction for every additional 24 hour period.
So, 1 hour late and 23 hours late both result in a 5\% point deduction, 25 hours late results in a 15\% point deduction, etc.

\section{Electronics}
Please turn off any and all pagers, cell-phones, and the like while you are in my class.
It is a common courtesy to your instructors and your fellow students to do the same while in any other class on campus.
Distraction is contagious.

\section{Communications}
Course materials, homework and project assignments, and resources will be posted to the course website.

I will send you messages concerning the class via e-mail.
So check your email regularly (at least once a day, and preferably twice).

\section{Academic Dishonesty}
Don't cheat! Don't even think about it.
If you cheat, you will be caught and the penalty is severe.
For a first offense, you will receive a score of zero on the assignment and your final grade will be reduced by one letter grade.
A second offense will result in you receiving a grade of F in the class.
All instances of academic dishonesty will be reported to your advisor and university administration per the university's policies.

I expect you to do your own work.
Do not work with others and turn in duplicates.
Do not turn in work that you did not write (this includes material copied from the internet or other sources).
You are free to ask questions of others and learn from your friends, but not to copy from them.

Do not let others copy from you.
This is also academic dishonesty and will result in consequences.
You are responsible for ensuring others do not copy from your work.

Page 21 of the \href{http://registrar.mst.edu/academicregs/index.html}{Student Academic Regulations Handbook} describes the standard of conduct for students and gives examples of academic dishonesty.

\section{Concerns}
I will do my best to address any concerns you have about the class.
You simply need to ask me.
My immediate supervisor is Dr. Daniel Tauritz. If there are any problems that I am unable to resolve for you relevant to this class, address your concerns to him.

\section{Disability Accomodations}
If you have a documented disability and anticipate needing accommodations in this course, you are strongly encouraged to meet with me early in the semester.
You will need to request that the Disability Services staff send a letter to me verifying your disability and specifying the accommodation you will need before I can arrange your accommodation.
Disability Support Services is located in 204 Norwood Hall.
Their phone number is (573) 341-4211 and their email is \href{mailto:dss@mst.edu}{dss@mst.edu}.

\section{Title IX}
Missouri University of Science and Technology is committed to the safety and well-being of all members of its community.
US Federal Law Title IX states that no member of the university community shall, on the basis of sex, be excluded from participation in, or be denied benefits of, or be subjected to discrimination under any education program or activity.
Furthermore, in accordance with Title IX guidelines from the US Office of Civil Rights, Missouri S\&T requires that all faculty and staff members report, to the Missouri S\&T Title IX Coordinator, any notice of sexual harassment, abuse, and/or violence (including personal relational abuse, relational/domestic violence, and stalking) disclosed through communication including but not limited to direct conversation, email, social media, classroom papers, and homework exercises. Missouri S\&T’s Title IX Coordinator is Vice Chancellor Shenethia Manuel.
Contact her directly (\href{mailto:manuels@mst.edu}{manuels@mst.edu}; (573) 341-4920; 113 Centennial Hall) to report Title IX violations.
To learn more about Title IX resources and reporting options (confidential and non-confidential) available to Missouri S\&T students, staff, and faculty, please visit \href{http://titleix.mst.edu}{http://titleix.mst.edu}.

\section{Classroom Egress Maps}
Familiarize yourself with classroom and building exits. Classroom egress maps are posted on-line at: \href{http://designconstruction.mst.edu/floorplan/}{http://designconstruction.mst.edu/floorplan/}.

\end{document}