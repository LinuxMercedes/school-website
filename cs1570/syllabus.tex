\documentclass{article}

\usepackage{hyperref}
\hypersetup{colorlinks=false,
allbordercolors={0 0 0},
pdfborderstyle={/S/U/W 1}
} % Set link colors throughout the document

\title{CS 1570: Intro to C++ \\ Fall 2016 Syllabus}
\date{\today}
\author{Nathan Jarus}

\begin{document}
\maketitle

\section{Instructor}
The course will be taught by \textbf{Nathan Jarus}. \\*
\textbf{E-mail}: \href{mailto:jarus@mst.edu}{jarus@mst.edu} \\*
\textbf{Course Website}: \href{https://www.mst.edu/~nmjxv3/cs1570}{https://www.mst.edu/~nmjxv3/cs1570} \\*
\textbf{Office}: TBD \\*
\textbf{Office Hours}: 14:30--16:00 Tuesday and Thursday\\*
\textbf{Text}: \textit{Absolute C++}, Walter Savitch, 4\textsuperscript{th} edition or later\\

\section{Attendance}
Attendance in class is mandatory.
You will be responsible for all the material presented in class and otherwise designated (such as reading material).
Come to class on time and be prepared.
If you miss class for some reason, it is your responsibility to find out what you've missed; it is not my responsibility.
If you miss more than 3 classes, I WILL drop you from the class.
(If you wish to drop the class, don't assume that I will do it automatically when you stop coming to class. Please bring me a drop slip to sign.)
Be responsible for your actions and/or inaction.

\section{Electronics}
Please turn off any and all pagers, cell-phones, and the like while you are in my class.
It is a common courtesy to your instructors and your fellow students to do the same while in any other class on campus.
Distraction is contagious.

If you are a member of an emergency response team (such as an ambulance crew or rural fire-fighter), please let me know.

\section{Topics}
You can see the schedule of topics we will cover on the course website.
This outline may change as the semester proceeds.
Additional material not in the text may be presented.
The material that we will cover in the text is roughly outlined as follows:
\begin{itemize}
	\item (5th week) TEST \#1: chapters 1, 2
	\item (10th week) TEST \#2: chapters 3, 4, 16.1, 6.1, 5, 9, 11.1
	\item (15th week) TEST \#3: chapters 5 and 9 cleanup, 12, 6.2, 7, 8 and other optional material
\end{itemize}

\section{Grades}
Your grade in this class will be determined by your performance on tests and regularly assigned homework.
Tests will be announced at least one week in advance and are listed on the course schedule.
There will be 3 tests and a final exam.
You may not make-up tests unless you have acceptable and verifiable reasons for missing class that day.
Acceptable excuses include such circumstances as `acts of God' (e.g.\ death in the family, being hit by a large truck on the way to class, etc) and exclude such non-excuses as a faulty alarm clock, drinking binges, having a grand piano fall on you from a 5th floor window (this never really happens -- only in the cartoons), etc.
Do not bother to ask me if you may take a test early because I won't let you, no matter what the reason.
Dates that classes are in session are well published; don't ask me if you can miss a class or test because you want to leave for vacation early.

Your final grade will be based on a straight scale (90\% - 100\%  A;  80\% - 89\%  B; etc.).

Your grade will be split between homeworks and exams like so:
\begin{itemize}
	\item 50\% -- Exams
	\item 40\% -- Homeworks 1 - 9
	\item 10\% -- Final project
\end{itemize}

Your tests will account for 50\% of your final average, the homework/programs up to the last project will account for 40\% of your final average, and the final programming project will account for 10\% of your final average.

If you score below 60\% on any test you need to discuss the situation with me in person. Don’t ignore this warning.

You will most probably learn much you didn't already know.
I will expect you to really learn it.
I am not going to expect you to regurgitate stock answers to stock questions.
You will be asked to demonstrate solid understanding of the material, applying your new-found knowledge in many different ways.

Homework assignments will be programming problems designed to hone and test your C++ and problem solving skills.
These programs will be your primary contribution towards preparing for the exams.
If you fail to submit any one of these, you will hurt yourself/your grade twice.
Make every effort to submit each one.
You will submit these programs electronically using the cssubmit tool.
Electronic submission does not include email.
I will be using Canvas this semester, but only for posting grades.
I will show you later in class how to submit your programs.
Late submission will be penalized according to the following schedule:

\begin{itemize}
	\item First Late:
		\begin{itemize}
			\item 10\% penalty for first 24 hours (1 day)
			\item 50\% penalty for second 24 hours (2 days)
			\item no credit thereafter.
		\end{itemize}
	\item Second Late:
		\begin{itemize}
			\item 50\% penalty for the first 24 hours (1 day)
			\item no credit thereafter.
		\end{itemize}
\end{itemize}

\section{Compiler}
I will be using the GNU C++ compiler to compile and execute your programs after you have submitted them to me.
Thus, you must make sure that your programs will compile using the GNU compiler before you submit.
You may develop your programs on any platform and using any compiler, but it had better ultimately compile on the GNU compiler.
Understand that there are subtle differences between compilers and you will be responsible for that problem if you use any compiler other than GNU compiler.
The machines you will use to submit your homework have the correct version of the GNU compiler installed.

\section{Communications}
I will send you messages concerning the class via e-mail.
So check your email regularly (at least once a day, and preferably twice).
Please note that I usually log off my computer and email at about 5 p.m.
If you send me an email after that time, I probably won't (but I might) read and reply to it until the next morning.
I usually don't check email on the weekend also.

Course materials, homework and project assignments, and resources will be posted to the course website.

\section{Academic Dishonesty}
Don't cheat! Don't even think about it.
If you cheat, you will be caught and the penalty is severe.
In response to a first offense, I will give you a zero for the assignment, lower your semester grade one letter, and notify your adviser and the Provost.
Your second offense will get you ejected from the class.

I expect you to do your own work.
This means that you should not work with another student on your programs; I want to see your work.
Do not work with others and turn in duplicates.
Do not turn in work that you did not write (this includes material copied from the internet or other sources).
Don't try to fool me.
You are free to ask questions of others and learn from your friends, but not to copy ideas and/or code.

Do not let others copy from you.
This is also academic dishonesty and will result in consequences.
You are responsible for ensuring others do not copy from your work.

Page 21 of the \href{http://registrar.mst.edu/academicregs/index.html}{Student Academic Regulations Handbook} describes the standard of conduct for students and gives examples of academic dishonesty.

\section{Help}
Be sure to seek my help if you need it.
I will be glad to help if I can; you only have to ask.
Check my office hours.
However, I am in my office most of the day and am happy to see you any time I can.
E-mail me for an appointment or ask me after class for an appointment.

Note that tutoring will also be provided; details will be posted on the course website when they are available.
Avail yourself of this service if you need to.

\section{Concerns}
I will do my best to address any concerns you have about the class.
You simply need to ask me.
My immediate supervisor is Mr. Clayton Price. If there are any problems that I am unable to resolve for you relevant to this class, address your concerns to him. His office is in CS 325G.

\section{Disability Accomodations}
If you have a documented disability and anticipate needing accommodations in this course, you are strongly encouraged to meet with me early in the semester.
You will need to request that the Disability Services staff send a letter to me verifying your disability and specifying the accommodation you will need before I can arrange your accommodation.
Disability Support Services is located in 204 Norwood Hall.
Their phone number is (573) 341-4211 and their email is \href{mailto:dss@mst.edu}{dss@mst.edu}.

\section{Title IX}
Missouri University of Science and Technology is committed to the safety and well-being of all members of its community.
US Federal Law Title IX states that no member of the university community shall, on the basis of sex, be excluded from participation in, or be denied benefits of, or be subjected to discrimination under any education program or activity.
Furthermore, in accordance with Title IX guidelines from the US Office of Civil Rights, Missouri S\&T requires that all faculty and staff members report, to the Missouri S\&T Title IX Coordinator, any notice of sexual harassment, abuse, and/or violence (including personal relational abuse, relational/domestic violence, and stalking) disclosed through communication including but not limited to direct conversation, email, social media, classroom papers, and homework exercises. Missouri S\&T’s Title IX Coordinator is Vice Chancellor Shenethia Manuel.
Contact her directly (\href{mailto:manuels@mst.edu}{manuels@mst.edu}; (573) 341-4920; 113 Centennial Hall) to report Title IX violations.
To learn more about Title IX resources and reporting options (confidential and non-confidential) available to Missouri S\&T students, staff, and faculty, please visit \href{http://titleix.mst.edu}{http://titleix.mst.edu}.

\section{Classroom Egress Maps}
Familiarize yourself with classroom and building exits. Classroom egress maps are posted on-line at: \href{http://designconstruction.mst.edu/floorplan/}{http://designconstruction.mst.edu/floorplan/}.

\end{document}